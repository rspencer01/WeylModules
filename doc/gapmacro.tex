%%%%%%%%%%%%%%%%%%%%%%%%%%%%%%%%%%%%%%%%%%%%%%%%%%%%%%%%%%%%%%%%%%%%%%%%%%%%%
%%
%A  gapmacro.tex                GAP manual                   Frank Celler
%A                                                           Heiko Theissen
%A                                                           Alexander Hulpke
%%
%%  @(#)$Id: gapmacro.tex,v 4.64.4.1 2005/08/26 08:59:57 gap Exp $
%%
%%  DO NOT RELY ON MACRO DEFINITIONS IN THIS FILE!
%%  The official definition of the manual style is to be found in the file
%%  `ext:document.tex'.
%%
%%  The following macros are defined in this file.
%%
%%  `text'   set text in typewriter style (use `\<' instead of `<')
%%  <text>   set text in italics (use $\<$ instead of $<$ for less than)
%%  *text*   set text in emphasized style (i.e. slanted)
%%  $a.b$    same as $a \cdot b$ (use $\.$ instead of $.$ for full stop)
%%  "ref"    refer to a label (like "function!for category")
%%  \pif     sets a single '
%%  \cite{.} make a citation
%%  \index{.} \indextt{.} make index entry (\indextt in typewriter style)
%%
%%  \beginitems         produce itemized texts with 3pc hanging indentation
%%    item & text
%%
%%    item & text ...
%%  \enditems
%%
%%  \begintt            verbatim text in typewriter style
%%    verbatim material
%%  \endtt
%%  \beginexample     verbatim text in typewriter style
%%    verbatim material
%%  \endexample
%%
%%  \Input{file}  includes file `file.tex'
%%  \Chapter{title} followed by a blank line
%%  \Section{title} followed by a blank line
%%      make  chapter  or section   title. Automatically  generates  table of
%%      contents. \null after \Section{...} inhibits indexing.
%%  \>function( arguments )!{ index subentry }
%%  \>`a binop b'{binary operation}!{ index subentry }
%%      make a  heading for a subsection   explaining a function  or a binary
%%      operation. This automatically generates   a label and an  index entry
%%      (with optional subentry).
%%  \){\fmark ...}
%%      the same without label and index entry
%%
%%  \URL{url}
%%  \Mailto{}
%%
%%  \BeginningOfBook
%%  \FrontMatter, \Chapters, \Appendices, \EndMatter     parts of the book
%%  \Bibliography, \Index, \TableOfContents  make these chapters (w/o head)
%%  \EndOfBook
%%

\input amssym.tex

% Page dimensions and double column output.
\hsize 39pc
\vsize 52pc

% do we run pdftex? 
\expandafter\ifx\csname pdfdest\endcsname\relax
\immediate\write16{Running TeX}
% no: define dummy bookmarking functions
\global\def\bookmarkdestin#1{}
\global\def\setbookmarkind#1#2{}
\global\def\setbookmark#1#2{}
\global\def\indexbookmark#1#2{}
\else
\immediate\write16{Running PDFTeX}
% yes: define macros to do pdf stuff and set some parameters
\global\def\bookmarkdestin#1{\pdfdest name {#1} xyz}
\global\def\setbookmarkind#1#2{%
  \setindent{\chapterlen{\the\chapno}}
  \pdfoutline goto name {#1} count -\indentno {#2}}
\global\def\setbookmark#1#2{%
  \setindent{\chapterlen{\the\chapno}}
  \pdfoutline goto name {#1} count 0 {#2}}
% special treatment for the index to get proper indention
\global\def\indexbookmark#1#2{%
  \pdfoutline goto name {#1} count -27 {#2}}
\pdfinfo{
/Subject (GAP Manual)
/Author (The GAP Group)
}
\pdfcatalog{
/URI (http://www.gap-system.org)
/PageMode /UseOutlines}
\pdfcompresslevel 9
\fi

%%%%%%%%%%%%%%%%%%%%%%%%%%%%%%%%%%%%%%%%%%%%%%%%%%%%%%%%%%%%%%%%%%%%%%%%%%%%
% 
%  Generic double column output.
%
%    Modified from a routine written by Donald Knuth (The TeXBook, App. E)
%
%%%%%%%%%%%%%%%%%%%%%%%%%%%%%%%%%%%%%%%%%%%%%%%%%%%%%%%%%%%%%%%%%%%%%%%%%%%%
%
%  The user may modify the following to his tastes:
%
%      \pagewidth     vertical length of page.
%      \pageheight    horizontal width of page.
%      \colwidth      column width
%      \separator     macro to generate column separator. Default is nothing.
%                     \rulesep sets it to \vrule. \norulesep doesn't.
%      \makepage      default is what is contained in plain.
\catcode`@=11 % from plain.tex
% Create and initialize new dimensions.
\newdimen\pagewidth  \newdimen\pageheight  \newdimen\colwidth
\pagewidth=\hsize    \pageheight=\vsize    \colwidth=3.2truein

\def\draft{\hsize 42pc\vsize 63pc\pagewidth=\hsize \pageheight=\vsize}

\def\pif{\char39}

% This routine is used by \output ; this is different from
%   the one found in App. E.
\def\onepageout#1{{\setbox255=\vbox{#1}
  \hsize=\pagewidth \vsize=\pageheight \plainoutput}}
\def\normaloutput{\onepageout{\unvbox255}}
\maxdeadcycles=100 % \output is called quite often

\output={\normaloutput}
\newbox\partialpage \newdimen\origvsize \newif\ifrigid
\def\begindoublecolumns{\global\origvsize=\vsize \begingroup
  \output={\global\setbox\partialpage=\vbox{\unvbox255\kern0pt}}\eject
  \output={\doublecolumnout} \hsize=\colwidth \dimen@=\pageheight
  \advance\dimen@ by-\ht\partialpage \multiply\dimen@ by2
  \ifdim\dimen@<2\baselineskip \dimen@=2\baselineskip\fi
  \vsize=\dimen@}
\def\enddoublecolumns{\output={\balancecolumns}\eject
  \endgroup \global\vsize=\origvsize \pagegoal=\vsize}
\def\doublecolumnout{\splittopskip=\topskip \splitmaxdepth=\maxdepth
  \setbox0=\vsplit255 to.46\vsize \setbox2=\vsplit255 to.46\vsize
  %\setbox0=\vbox{A\unvbox0B\vfill}\setbox2=\vbox{C\unvbox2D\vfill}%
  \onepageout\pagesofar \global\vsize=2\pageheight
  \unvbox255 \penalty\outputpenalty}
\def\pagesofar{\unvbox\partialpage
  \wd0=\hsize \wd2=\hsize
  %\hbox to\pagewidth{\box0\hfil\separator\hfil\box2}}
  \hbox to\pagewidth{\valign{##\vfill\cr%
  \vbox{\unvbox0}\cr\noalign{\hfil\separator\hfil}\vbox{\unvbox2}\cr}}}
\def\norulesep{\let\separator=\relax}
\def\rulesep{\let\separator=\vrule}
\let\separator=\relax
\def\balancecolumns{\setbox0=\vbox{\unvbox255} \dimen@=\ht0
  \advance\dimen@ by\topskip \advance\dimen@ by-\baselineskip
  \divide\dimen@ by2 \splittopskip=\topskip
  {\vbadness=10000 \loop \global\setbox3=\copy0
    \global\setbox1=\vsplit3 to\dimen@
    \ifdim\ht3>\dimen@ \global\advance\dimen@ by1pt \repeat}
  \ifrigid
    \setbox0=\vtop{\unvbox1}
    \setbox2=\vtop{\unvbox3}
  \else
    \setbox0=\vbox to\dimen@{\unvbox1}
    \setbox2=\vbox to\dimen@{\dimen2=\dp3 \unvbox3\kern-\dimen2 \vfil}
  \fi
  \global\vsize=\origvsize \pagesofar}
\catcode`@=12
%%%%%%%%%%%%%%%%%%%%%%%%%%%%%%%%%%%%%%%%%%%%%%%%%%%%%%%%%%%%%%%%%%%%%%%%%%%%

\colwidth 19pc
\newdimen\manindent      \manindent 3pc
\newdimen\smallmanindent \smallmanindent 1pc
\parskip 1ex plus 0.5ex minus 0.5ex
\parindent 0pt

% Additional fonts.
\font\inchhigh=cminch
\font\titlefont=cmssdc10 at 40pt
\font\subtitlefont=cmssdc10 at 24pt
\font\secfont=cmssdc10 at 14pt
\font\sf=cmss10
\font\bsf=cmssdc10
\font\smallrom=cmr8
\font\sevenit=cmti10 at 7pt \scriptfont\itfam=\sevenit
\font\fiveit=cmti10 at 5pt  \scriptscriptfont\itfam=\fiveit

% If you don't have `msb' fonts, replace the next 4 lines by `\let\Bbb=\bf'.
\newfam\msbfam \def\Bbb{\fam\msbfam}
\font\tenmsb=msbm10         \textfont\msbfam=\tenmsb
\font\sevenmsb=msbm7        \scriptfont\msbfam=\sevenmsb
\font\fivemsb=msbm5         \scriptscriptfont\msbfam=\fivemsb
\font\sevenmsa=msam7

% the dark triangle
\def\darktriangleright{\raise.4ex\hbox{\sevenmsa\char"49}}

% Math mode should use text italic.
{\count0=\itfam \advance\count0 by-1 \multiply\count0 by"100
 \count1=`A
 \loop \count2=\mathcode\count1 \advance\count2 by\count0
       \global\mathcode\count1=\count2
      {\advance\count1 by'040
       \count2=\mathcode\count1 \advance\count2 by\count0
       \global\mathcode\count1=\count2}
 \ifnum\count1<`Z \advance\count1 by1\repeat}

% macros for verbatim scanning (almost copied from `The TeXbook')
\chardef\other=12
\def\undocatcodespecials{\catcode`\\=\other     \catcode`\{=\other
  \catcode`\}=\other     \catcode`\<=\other     \catcode`\$=\other
  \catcode`\%=\other     \catcode`\~=\other     \catcode`\^=\other
  \catcode`\_=\other     \catcode`\*=\other     \catcode`\`=\other
  \catcode`\!=\other     \catcode`\"=\other     \catcode`\&=\other
  \catcode`\#=\other     \catcode`\|=\other}
\def\ttindent{5mm} % indentation amount of verbatim examples
\def\kernttindent{\hskip\ttindent\relax} % for use in %\begintt
{\obeyspaces\global\let =\ }
{\obeylines\gdef\obeylines{\catcode`^^M=\active}\gdef^^M{\par}%
\catcode`#=\active \catcode`&=6 \gdef#{\char35}%
%\catcode`#=\active \catcode`&=6 \gdef#&1^^M{\hbox{\char35 &1}^^M}%
 \gdef\ttverbatim{\begingroup\undocatcodespecials \catcode`\#=\active%
   \parindent 0pt \def\_^^M{\allowbreak}\def\${$}\def\`{`}%
   \def\par{\ifvmode\allowbreak\vskip 1pc plus 1pt\else\endgraf\penalty100\relax\fi}%
   \obeyspaces \obeylines \tt}}
\outer\def\begintt{\par
  \begingroup\advance\leftskip by \ttindent
  \ttverbatim \parskip=0pt \catcode`\|=0 \rightskip-5pc \ttfinish}
{\catcode`\|=0 |catcode`|\=\other % | is temporary escape character
  |obeylines % end of line is active
  |gdef||{|char124} %
  |gdef|ttfinish#1^^M#2\endtt{#1|medskip{#2}|endgroup %
  |endgroup%
  |vskip-|parskip|medskip|noindent|ignorespaces}}
\def\unstableoutput{}
\outer\def\beginexample{\par
  \begingroup\advance\leftskip by \ttindent
  \ttverbatim \parskip=0pt \catcode`\|=0 \rightskip-5pc \examplefinish}
{\catcode`\|=0 |catcode`|\=\other % | is temporary escape character
  |obeylines % end of line is active
  |gdef||{|char124} %
  |gdef|examplefinish#1^^M#2\endexample{#1|medskip{#2}|endgroup %
  |endgroup%
  |vskip-|parskip|medskip|noindent|ignorespaces}}

% Input/output streams. Chapter and section counters.
\newwrite\labelout \newwrite\indexout \newwrite\secindout
\newwrite\tocout   \newwrite\citeout  \newwrite\ans
\newread \labelin  \newread \indexin  \newread \tocin  \newread \citein
\countdef\chapno=1 \newcount\secno    \newcount\subsecno \newcount\exno
\newcount\indentno 
\newcount\chapnum  %GG \chapnum is a numerical \chapno for the .six file
\def\chapterno{{\edef\tempa{\thechapter}\tempa}}
%\def\folio{\ifnum\pageno<0 \romannumeral-\pageno \else
%  \chapterno\ifx\thechapter\emptychapter\else--\fi \number\pageno\fi}
%\def\doindex#1#2#3{\write\indexout{\noexpand\indexentry{#1#2#3}%
%  {\ifnum\pageno<0 \romannumeral-\pageno \else
%   \thechapter\ifx\thechapter\emptychapter\else--\fi \number\pageno\fi}}%
%  \ifvmode\nobreak\fi}

%AH
\def\folio{\ifnum\pageno<0 \romannumeral-\pageno \else
  \number\pageno\fi}
  
%\def\doindex#1#2#3{\protectedwrite\indexout{\noexpand\indexentry{#1#2#3}%
%  {\ifnum\pageno<0\romannumeral-\pageno\else%
%   \number\pageno\fi}}%
%  \ifvmode\nobreak\else\unskip\fi}
{\catcode`@=11 
\catcode`<=\active \catcode`_=\active \catcode`!=\active \catcode`*=\active 
\catcode``=\active 

\global\let\n@exp\noexpand

\gdef\doindex#1#2#3{\begingroup
  %\let_\underscore\let`\lq\let<\<\let!\excl
  \def\noexpand{}% this kills noexpands from some index entries
  \def\protect{\n@exp\n@exp\n@exp}%
  \def_{\protect_}\def*{\protect*}\def`{\protect `}\def<{\protect <}\def!{\protect!}%
  \edef\tmp{\n@exp\n@exp\n@exp\indexentry{#1#2#3}}%
  \expandafter\write\expandafter\indexout\expandafter{\tmp{\folio}}%
  %{\scrollmode\errorcontextlines 1\show\tmp\errorstopmode}%
  \ifvmode\nobreak\else\unskip\fi\endgroup}%
}

% Additional active characters and their default meanings.
\mathcode`.="2201 \mathchardef\.="702E
\def\undoquotes{\catcode`'=12 \catcode``=12 \def\"##1{{\accent127 ##1}}}
\def\excl{!} \chardef\lqq=`\\ \let\underscore=\_
\chardef\<=`<
\catcode`!=\active \let!=\excl
\catcode`^=\active \def^{\ifmmode\sp\else{\char`\^}\fi}
\catcode`_=\active \def\activeusc{\ifmmode\sb\else\_\fi}
                   \let_=\activeusc                     \let\_=\underscore
\catcode`*=\active \def*{\ifmmode\let\next=\*\else\let\next=\bold\fi\next}
                   \def\bold#1*{{\bf #1\/}}             \chardef\*=`*
\catcode`<=\active \def<#1>{{\chardef*=`*\let_=\_\it#1\/}}
                                                       
\catcode`"=\active \def"{\begingroup\undoquotes\doref}  \chardef\"=`"
                                                        \chardef\\=`\\

{\catcode`|=0 \catcode`\\=12 |gdef|bs{\}}
% Labels (which are automatically generated by ``\Section'' and ``\>'').
%\newif\iflabmultdef
\newif\iflabundef
\newif\iflabchanged
\def\doref#1"{\bookref#1:"}
\def\bookref#1:#2"{\def\tempa{#2}\ifx\tempa\empty\printref{\book:#1}\else
  \printbookref#1:#2"\fi}
\def\printbookref#1:#2:"{\printref{#1:#2}}

% Macros which write labels, citations and index entries on auxiliary files.
% 
% 1. Some code was moved around so that all the labels and indexing stuff
%    would be together.
% 2. Modified \makelabel, \printref and \label to be more like the LaTeX
%    commands \newlabel, \ref and \label, respectively. The commands:
%    \@xp, \@firstoftwo, \@secondoftwo, \namedef, \@ifundefined 
%    \protectedwrite are adapted pieces of LaTeX code (\@xp is actually
%    AmSLaTeX's abbreviation of \expandafter).
% 3. The upshot of 2. is that now underscore is allowed in labels and
%    index-entries and that the \makelabel commands written to the
%    manual.lab files have arguments that are no longer put in lowercase.
%    The labels formed inside TeX from the \makelabel commands maps
%    underscores to | (needed to be a non-active character) and then
%    converts to lowercase.
% 4. Since GapDoc uses < and > in some of its labels a temporary re-defintion
%    of <#1> was necessary, inside \@xplowr.
% - GG 2001/06/11, 2001/09/02
\catcode`@=11
\let\@xp\expandafter
\long\def\@firstoftwo#1#2{#1}
\long\def\@secondoftwo#1#2{#2}
\def\namedef#1{\expandafter\def\csname #1\endcsname}
\def\@ifundefined#1{\@xplowr{#1}%
  \@xp\ifx\csname \xpandlowr\endcsname\relax%
    \@xp\@firstoftwo%
  \else%
    \@xp\@secondoftwo%
  \fi}
\def\protectedwrite#1#2{\begingroup\let_\relax\immediate\write#1{#2}\endgroup}
\def\@xplowr#1{\begingroup\def_{|}\def<##1>{|##1|}\let\next=\relax%
  \expandafter\edef\next{\gdef\noexpand\xpandlowr{#1}}%
  \lowercase\expandafter{\next}\endgroup}
\def\makelabel#1#2{\@xplowr{r@#1}%
  \@ifundefined{\xpandlowr}%
    \relax%
    {\let_\activeusc}%\protectedwrite{16}{Label `#1' multiply defined.}}%
  \let_\activeusc%\protectedwrite{16}{Defined label `#1'.}%
  \begingroup\edef_{|}\global\namedef{\xpandlowr}{#2}\endgroup}
\gdef\printref#1{\@xplowr{#1}%
  \@xp\ifx\csname r@\xpandlowr\endcsname\relax\lqq\xpandlowr''%
  \protectedwrite{16}{Label `#1' undefined.}\global\labundeftrue%
  \else\csname r@\xpandlowr\endcsname\fi\endgroup\let_\activeusc}

% Macros which write labels, citations and index entries on auxiliary files.
\gdef\label#1{\@xplowr{#1}
  \ifnum\secno=0 \edef\@currentlabel{\thechapter}\else
  \ifnum\subsecno=0 \edef\@currentlabel{\thechapter.\the\secno}\else
  \edef\@currentlabel{\thechapter.\the\secno.\the\subsecno}\fi\fi
  \@xp\ifx\csname r@\book:\xpandlowr\endcsname\@currentlabel\else%
  %\immediate\write16{Label `\book:#1' has changed.}
  \global\labchangedtrue\fi
  \protectedwrite\labelout{\string\makelabel{\book:#1}{\@currentlabel}}%
  \let_\activeusc}

 \gdef\sigel#1{[\expandafter\ifx\csname c@#1\endcsname\relax
  \immediate\write16{Reference `#1' undefined.}\global\labundeftrue
  #1\else \csname c@#1\endcsname\fi]}
 \gdef\bibitem[#1]#2{\expandafter\gdef\csname c@#2\endcsname{#1}%
  \item{\sigel{#2}}%
  \immediate\write\labelout{\noexpand\setcitlab{#2}\noexpand{#1}}}
 \gdef\setcitlab#1#2{\expandafter\gdef\csname c@#1\endcsname{#2}}
\catcode`@=12%

\def\cite#1{\write\citeout{\bs citation{#1}}\sigel{#1}}
\def\dosecindex#1#2#3{\ifchapter{\let\ =\space%
   \protectedwrite\secindout{#1 \the\chapno.\the\secno. #2#3}}\fi}
\def\bothindex#1#2#3#4{\doindex{#2}{#3}{#4}\dosecindex{#1}{#2}{#4}}
\def\index#1{\bothindex I{#1}{}{}}
\def\atindex#1#2{\doindex{\scrubafterexcl#1!\end}{#2}{}\dosecindex I{#1}{}}
\def\indextt#1{\doindex{\scrubafterexcl#1!\end}%
   {@`\scrubafterexcl#1!\end'\scrubbeforeexcl#1!\end}{}\dosecindex I{#1}{}}
%\def\indextt#1{\atindex{#1}{@`#1'}}
\def\indexit#1{{\it #1}}

% Macros for generating the table of contents.
\newif\iffirstsec \firstsectrue
\def\dotsfill{\leaders\hbox to12pt{\hss.\hss}\hfill}

\def\appcontents#1#2#3{}

\def\chapcontents#1#2#3{%
   %\iffirstsec\firstsecfalse\else\line{}\fi% empty line
   \par\penalty-5\medskip
   \line{\bf\kern\manindent\vbox{\advance\hsize by-\manindent
   \advance\hsize by-1.5em
   \rightskip 0pt plus1fil \emergencystretch 3em
   \noindent\llap{\hbox to\manindent{\hss #1\kern\smallmanindent}}\strut
   #2~\hfill \strut\rlap{\hbox to1.5em{\hss #3}}}\hfil}}

\def\seccontents#1#2#3{
   \par\penalty-5\medskip
   \setchapterlen#1.
   \line{\kern\manindent\vbox{\advance\hsize by-\manindent
   \advance\hsize by-1.5em
   \rightskip 0pt plus1fil \emergencystretch 3em
   \noindent\llap{\hbox to\manindent{\hss #1\kern\smallmanindent}}\strut
   #2~\dotsfill \strut\rlap{\hbox to1.5em{\hss #3}}}\hfil}}

% GG ... the idea of \ors is to put the right no. of \or s in the \appno
%        macro so that for an appendix:
%         1. \the\chapno gives the number needed in the .six file, and
%         2. \appno\chapno gives the right letter A, B, ... for the .lab file
\newtoks\ortoks
\long\def\addor{\begingroup\ortoks\expandafter{\ors\or}
   \xdef\ors{\the\ortoks}\endgroup}
\def\setors{{\xdef\ors{} \count0=\chapno
   \loop\ifnum\count0>0 \addor \advance\count0 by -1\repeat}}
\def\appno#1{\expandafter\ifcase\expandafter#1\ors\or A\or B\or C\or D\or E\or 
   F\or G\or H\or I\or J\or K\or L\or M\or N\or O\or P\or Q\or R\or S\or T\or 
   U\or V\or W\or X\or Y\or Z\else\immediate\write16{Counter too large} \fi}

% Macros for chapter and section headings.
\def\filename{appendix}
\def\tocstrut{{\setbox0=\hbox{1}\vrule width 0pt height\ht0}}
\outer\def\Input#1{\def\filename{#1.tex}\input #1}
\def\emptychapter{\noexpand\tocstrut}
\def\normalchapter{\the\chapno}
% GG 
\def\appendixchapter{\appno\chapno} % \appno behaves like LaTeX's \Alph
\newif\ifchapter           % set to true when \thechapter = \normalchapter
                           % ... or ...  when \thechapter = \appendixchapter

{\catcode`@=11%
%
\gdef\setchapterlen#1.#2.{\expandafter\gdef\csname ch@#1\endcsname{#2}}
\gdef\chapterlen#1{%
\ifchapter
\expandafter\ifx\csname ch@#1\endcsname\relax%
0%
\else\csname ch@#1\endcsname\fi
\else 0\fi}%
}
\def\setindent#1{\indentno=#1}


\def\Chapter#1 \par{\vfill\supereject \headlinefalse
%  \ifodd\pageno\else\null\vfill\eject\headlinefalse\fi
  \advance\chapno by1 \secno=0\subsecno=0\exno=0%\ifnum\pageno>0 \pageno=1 \fi
  \bookmarkdestin{#1}
  \def\chapname{#1} \label{#1}
  \immediate\write16{Chapter `#1' .}%
  \write\tocout{\noexpand\chapcontents{\thechapter}{#1}{\the\pageno}}
  \ifchapter
    \immediate\write\secindout{C \filename\space\the\chapno. \chapname}\fi
  \setbookmarkind{#1}{#1}
  \setbox0=\hbox{\inchhigh\kern-.075em \chapterno}
  \setbox1=\vbox{\titlefont \advance\hsize by-\wd0 \advance\hsize by-1em
%    \hyphenpenalty=10000 \linepenalty=10000
    \leftskip 0pt plus\hsize \parfillskip 0pt \baselineskip 44pt\relax #1}
  \line{\box0\hfil\box1}\nobreak \mark{}\vskip 40pt}

% start hack added by TB
\def\PreliminaryChapter#1 \par{\vfill\supereject \headlinefalse
%  \ifodd\pageno\else\null\vfill\eject\headlinefalse\fi
  \advance\chapno by1 \secno=0\subsecno=0\exno=0%\ifnum\pageno>0 \pageno=1 \fi
  \bookmarkdestin{#1}
  \def\chapname{#1 (preliminary)} \label{#1}
  \write\tocout{\noexpand\chapcontents{\thechapter}{#1 (preliminary)}{\the\pageno}}
  \ifchapter
    \immediate\write\secindout{C \filename\space\the\chapno. \chapname}\fi
  \setbookmarkind{#1}{#1}
  \setbox0=\hbox{\inchhigh\kern-.075em \chapterno}
  \setbox1=\vbox{\titlefont \advance\hsize by-\wd0 \advance\hsize by-1em
%    \hyphenpenalty=10000 \linepenalty=10000
    \leftskip 0pt plus\hsize \parfillskip 0pt \baselineskip 44pt\relax #1 (preliminary)}
  \line{\box0\hfil\box1}\nobreak \mark{}\vskip 40pt}
% end hack added by TB

\outer\def\Section#1#2\par{\bigbreak \advance\secno by1
  \subsecno=0
  \ifx\thechapter\emptychapter
    \edef\tempa{\the\secno}
  \else
    \edef\tempa{\thechapter.\the\secno}%
  \fi%
  \bookmarkdestin{\tempa}
  \setbookmark{\tempa}{#1}
  \expandafter\writesecline\tempa\\{#1}
  \dosecindex S{#1}{}
  \ifx#2\nolabel\else{\let\ =\space\label{#1}}\fi
  \ifx#2\null\else \edef\tempb{{#1@#1}}
    \expandafter\doindex\tempb{|indexit}{}\fi
  \noindent{\baselineskip 18pt\let!=\space \mark{Section \the\secno. #1}%
  \secfont \tempa \enspace #1}\par\nobreak\medskip}

\def\writesecline#1\\#2{\write\tocout{\noexpand\seccontents{#1}{#2}
  {\the\pageno}}}
\def\letter#1{\medskip{\secfont #1}\endgraf\nobreak%
  \edef\tempa{LeTtEr#1}%
  \bookmarkdestin{\tempa}%
  \setbookmark{\tempa}{#1}%
  }

% Macros for generating paragraph headings (e.g., function descriptions).
\def\fmark{\noindent\llap{\darktriangleright\rm\enspace}}
\def\moveup#1{\leavevmode \raise.16ex\hbox{\rm #1}}
\def\fpar{\endgraf\endgroup\nobreak\smallskip\noindent\ignorespaces}
\def\>{\begingroup\undoquotes\obeylines\angle}
\def\){\begingroup\obeylines\cloparen}
{\obeylines
\gdef\angle#1
  {\endgroup \ifx\par\fpar \global\def\susemarker{\fmark}\else%
  \global\advance\subsecno by1%
  \global\def\susemarker{\noindent\llap{\smallrom\the\subsecno\kern.5ex%
  \darktriangleright\rm\enspace}}%
    \ifvmode \vskip -\lastskip \fi \medskip%
    \begingroup\let\par=\fpar \parskip 0pt \fi%
  \endgraf\nobreak\oporfunc#1\end}%
\gdef\cloparen#1
  {\endgroup \ifx\par\fpar \else%
    \ifvmode \vskip -\lastskip \fi \medskip%
    \begingroup\let\par=\fpar \parskip 0pt \fi%
  \endgraf{\def\[{\moveup\lbrack}\def\]{\moveup\rbrack}\def\|{\vrule\relax}%
  \noindent\typewriter#1'}}%
\gdef\scanparen#1(#2\end{\def\tempa{#2}\ifx\tempa\empty%
  \def\next{\begingroup\cloparen\susemarker#1
  \label{#1}\bothindex F{#1}{@`#1'}{}}%
  \else\def\next{\delparen#1(#2\end}\fi \next}}
\def\delparen#1(\end{\function#1}
\def\oporfunc#1#2\end{\ifx#1`\def\next{\oporvalue#1#2}\else
                             \def\next{\scanparen#1#2(\end}\fi \next}
\long\def\oporvalue`#1'#2{\ifx#2V\def\next{\operation`#1'{#1}@{`#1'} V}\else
                                 \def\next{\operation`#1'{#2}}\fi \next}
\long\def\operation`#1'#2#3{{\def\[{\moveup\lbrack}\def\]{\moveup\rbrack}%
  \def\|{\vrule\relax}}
  \susemarker\typewriter#1'%
  \ifx#3!\def\next{\suboperation{#2}}
    \else\ifx#3@\def\next{\subatoperation{#2}}
    \else\endheaderline \label{#2}%
    \bothindex F{#2}{}{}\let\next=#3\fi\fi\next}
\long\def\function#1(#2)#3{{\def\[{\moveup\lbrack}\def\]{\moveup\rbrack}%
  \def\|{\vrule\relax}
  \susemarker\typewriter#1(#2)'}%
  \ifx#3!\def\next{\subfunction{#1}}\else
    \endheaderline\label{#1}\bothindex F{#1}{@`#1'}{}%
    \let\next=#3\fi\next}
\def\subfunction#1#2{\endheaderline\label{#1!#2}%
  \bothindex F{#1}{@`#1'}{!#2}}
\def\suboperation#1#2{\endheaderline\label{#1!#2}%
  \bothindex F{#1}{}{!#2}}
\def\subatoperation#1#2{\endheaderline\label{#1}%
  \doindex{\scrubafterexcl#1!\end}{@#2}{}
  \dosecindex F{#1}{}}
\def\scrubafterexcl#1!#2\end{#1}
\def\scrubbeforeexcl#1!#2\end{\if\relax#2\else!\removeendexcl#2\fi}
\def\removeendexcl#1!{#1}

\def\endheaderline{\hskip 0pt plus 1filll}

% Macro for item lists.
\catcode`&=\active
\def\beginitems{%
  \smallskip
  \begingroup
    \def&{\par \nobreak \hangindent\manindent \hangafter 0
      {\parskip 0pt\noindent}\ignorespaces}
    \parindent 0pt
    \catcode`&=\active
}
\def\enditems{\par \endgroup \smallskip \noindent \ignorespaces}

% Macro for item lists.
\def\beginlist{%
  \smallskip
  \begingroup
    \parindent=2em
}
\def\endlist{\par \endgroup \smallskip \noindent \ignorespaces}
\catcode`&=4

% Macros for exercises.
\outer\def\exercise{\advance\exno by1\begingroup
  \def\par{\endgraf\endgroup\medskip\noindent}
  \medskip\noindent{\bf Exercise \chapterno.\the\exno.}\quad}
\outer\def\answer{\immediate\write\ans{}%
  \immediate\write\ans{\noexpand\answerto{\thechapter.\the\exno.}}%
  \copytoblankline}
\def\answerto#1{{\noindent\bf #1}}
\def\copytoblankline{\begingroup\setupcopy\copyans}
{\undoquotes
\gdef\setupcopy{\undocatcodespecials \obeylines \obeyspaces}
\obeylines \gdef\copyans#1
  {\def\next{#1}%
  \ifx\next\empty\let\next=\endgroup %
  \else\immediate\write\ans{\next}\let\next=\copyans\fi\next}}

% Macros for the active backquote character (`).
{\catcode`.=\active \gdef.{\char'056 \penalty0}}
\def\writetyper{\catcode`.=\active \chardef\{ =`{ \chardef\}=`}
                      \chardef*=`* \chardef"=`"   \chardef~=`~}
\catcode``=\active
\def`{\futurelet\next\backquote}
\def\typewriter#1'{\leavevmode{\writetyper \chardef`=96 \tt #1}}
\def\backquote{\ifx\next`\let\next=\doublebackquote
                    \else\let\next=\typewriter \fi \next}
\def\doublebackquote`{\lqq}

% The \Package command for argument <pkg> defines a macro \<pkg> that
% sets the text <pkg> in sans-serif i.e. a share package author should
% put a \Package{<pkg>} in their manual.tex file and then use {\<pkg>}
% when they refer to package <pkg>.

\def\Package#1{\namedef{#1}{{\sf #1}}}
\Package{GAP}
\Package{MOC}
\Package{ATLAS}

% For one-off references to a share package there is the following:

\def\package#1{{\sf #1}}

% The day

\def\Day{\number\day}

% The month in words

\def\Month{\ifcase\month\or January\or February\or March\or April\or May\or
  June\or July\or August\or September\or October\or November\or December\fi}

% The year

\def\Year{\number\year}

% The date

\def\Today{{\Day} {\Month} {\Year}}
  
% Miscellaneous macros.
\def\N{{\Bbb N}} \def\Z{{\Bbb Z}} \def\Q{{\Bbb Q}} \def\R{{\Bbb R}}
\def\C{{\Bbb C}} \def\F{{\Bbb F}} \def\calR{{\cal R}}

%T do we want these
%\def\stars{\bigskip\centerline{\*\qquad\*\qquad\*}\bigskip}

% Page numbers and running heads.
\newif\ifheadline
\nopagenumbers
\def\makeheadline{\vbox to0pt{\vskip-22.5pt\hbox to\pagewidth{\vbox to8.5pt
  {}\the\headline}\vss}\nointerlineskip}
\headline={\ifheadline\ifodd\pageno \righthead\hfil{\rm\folio}\else
                                    {\rm\folio}\hfil\lefthead \fi
  \else\global\headlinetrue \hfil\fi}

% Macro for inputting an auxiliary file.
\def\inputaux#1#2#3{\immediate\openin#1=#2\jobname.#3
  \ifeof#1\immediate\write16{No file #2\jobname.#3.}\else
  \immediate\closein#1 \input#2\jobname.#3 \fi}

% Macros for the parts of the manual.
\outer\def\FrontMatter{%
  \let\thechapter=\emptychapter
  \def\lefthead{\it\chapname} \let\righthead=\lefthead

  \begingroup
  \undoquotes
  \inputaux\labelin{}{lab}
  %\setbox0=\vbox{\Bibliography}
  \endgroup
  \labchangedfalse
  
  % Open the auxiliary files for output.
  \immediate\openout\tocout   =\jobname.toc
  \immediate\openout\labelout =\jobname.lab
  \immediate\openout\indexout =\jobname.idx
  \immediate\openout\secindout=\jobname.six
  \immediate\openout\citeout  =\jobname.aux
%  \immediate\openout\ans=answers
  \immediate\write\citeout{\bs bibstyle{alpha}}

  \ifodd\pageno\else\headlinefalse\null\vfill\eject\fi
%  \pageno=1
}
  
\outer\def\Chapters{\vfill\eject
  \chapno=0 \let\thechapter=\normalchapter \chaptertrue
  \def\lefthead{{\it Chapter \the\chapno. \chapname}}
  \def\righthead{\ifx\botmark\empty\lefthead\else{\it \botmark}\fi}}

% GG
% Rather than reset \chapno in \Appendices, for the .six file (which GAP's 
% help uses), we number appendices as if they were chapters (via \the\chapno) 
% i.e. if the last chapter was numbered 7 then for the .six file the first 
% appendix would be numbered 8.
% The \setors macro puts (\number\chapno - 1) \or s in the \ors macro so that 
% \appno\chapno (= \appendixchapter) numbers the appendices sequentially from A
% for the .lab file (which TeX uses).
\outer\def\Appendices{\vfill\eject
  \setors \let\thechapter=\appendixchapter \chaptertrue
  \def\lefthead{{\it Appendix \thechapter. \chapname}}
  \def\righthead{\ifx\botmark\empty\lefthead\else{\it \botmark}\fi}}

\def\EndMatter{\vfill\eject
  \def\thechapter{} \chapterfalse
  \def\lefthead{{\it \chapname}}
  \let\righthead=\lefthead
  \hbadness=5000\relax
  \gdef\EndMatter{}}

%%%%%%%%%%%%%%%%%%%%%%%%%%%%%%%%%%%%%%%%%%%%%%%%%%%%%%%%%%%%%%%%%%%%%%%%%%%%%
%%
%F  \BeginningOfBook  . . . . . . . . . . . . . . . . . . . .  start the book
%%
\def\BeginningOfBook#1{%
  \def\book{#1}%
  \pageno=-1%
  \pageno=1%
  \headlinefalse%
  \let\thechapter=\emptychapter%
  \def\lefthead{\it\chapname}%
  \let\righthead=\lefthead%
}
%
%
%%%%%%%%%%%%%%%%%%%%%%%%%%%%%%%%%%%%%%%%%%%%%%%%%%%%%%%%%%%%%%%%%%%%%%%%%%%%%
%%
%F  \UseReferences{<book-path>}	. . . use references from book in <book-path>
%%
\def\UseReferences#1{{\undoquotes
    \inputaux\labelin{#1/}{lab}}
}
\let\UseGapDocReferences\UseReferences %needed so that the HTML converter
                                       %can tell the difference
%
%
%%%%%%%%%%%%%%%%%%%%%%%%%%%%%%%%%%%%%%%%%%%%%%%%%%%%%%%%%%%%%%%%%%%%%%%%%%%%%
%%
%F  \EndOfBook	. . . . . . . . . . . . . . . . . . . . . . . .  end the book
%%
\def\EndOfBook{\vfill\supereject
  \immediate\write16{##}
  \immediate\closeout\citeout
  \immediate\write16{## Citations for BibTeX written on \jobname.aux.}
  \immediate\closeout\indexout
  \immediate\write16{## Index entries for makeindex written on \jobname.idx.}
  \immediate\closeout\secindout
  \immediate\write16{## Section index entries written on \jobname.six.}
  \immediate\closeout\labelout
  \immediate\write16{## Label definitions written on \jobname.lab.}
  \immediate\closeout\tocout
  \immediate\write16{## Table of contents written on \jobname.toc.}
  %\iflabmultdef\immediate\write16{## There were multiply-defined labels.}\fi
  \iflabundef\immediate\write16{## There were undefined labels or
  references.}\fi
  \iflabchanged\immediate\write16{## Labels have changed, run again. (Or
  they were multiply defined.)}\fi
  \immediate\write16{##}
  \end
}
%
%
%%%%%%%%%%%%%%%%%%%%%%%%%%%%%%%%%%%%%%%%%%%%%%%%%%%%%%%%%%%%%%%%%%%%%%%%%%%%%
%%
%F  \TableOfContents  . . . . . . . . . . . . . . produce a table of contents
%%
% name of the chapter containing the table of contents
\def\TOCHeader{Contents}

% explanation at the beginning of the table of contents
\def\TOCMatter{%
}

% macros for generating the table of contents
%\newif\iffirstsec\firstsectrue
\def\dotsfill{\leaders\hbox to12pt{\hss.\hss}\hfill}

% produce the chapter "Contents"
\outer\def\TableOfContents{\Chapter{\TOCHeader}

\TOCMatter
\vskip 20pt
\begingroup
\rigidfalse
\let!=\space
\begindoublecolumns
\inputaux\tocin{}{toc}\vfill\eject
\enddoublecolumns
\endgroup
}

% a one column version of \TableOfContents for short manuals
\outer\def\OneColumnTableOfContents{\Chapter{\TOCHeader}

\TOCMatter
\vskip 20pt
\begingroup
\rigidfalse
\let!=\space
\inputaux\tocin{}{toc}\vfill\eject
\endgroup
}

%%%%%%%%%%%%%%%%%%%%%%%%%%%%%%%%%%%%%%%%%%%%%%%%%%%%%%%%%%%%%%%%%%%%%%%%%%%%%
%%
%F  \TitlePage{<text>}	. . . . . . . . . . . . . . . . generate a title page
%%
\long\def\TitlePage#1{%
  \null\vfill#1\null\vfill\eject
}

%%%%%%%%%%%%%%%%%%%%%%%%%%%%%%%%%%%%%%%%%%%%%%%%%%%%%%%%%%%%%%%%%%%%%%%%%%%%%
%%
%F  \Colophon{<text>}	. . . . . . . . . . . . . . generate a colophon page
%%
\long\def\Colophon#1{\Chapter{}

#1\null\vfill\eject
}


%%%%%%%%%%%%%%%%%%%%%%%%%%%%%%%%%%%%%%%%%%%%%%%%%%%%%%%%%%%%%%%%%%%%%%%%%%%%%
%%
%F  \Answers  . . . . . . . . . . . . .  produce the answers to the exercises
%%
% header for the answers
\def\AnswersHeader{Answers to the Exercises}
%
% produce the chapter "Answers"
\def\Answers{\Chapter{\AnswersHeader}

  \parindent\manindent
  \parskip 1ex plus 0.5ex minus 0.5ex
  \immediate\closeout\ans
  \input answers
}

%%%%%%%%%%%%%%%%%%%%%%%%%%%%%%%%%%%%%%%%%%%%%%%%%%%%%%%%%%%%%%%%%%%%%%%%%%%%%
%%
%F  \Bibliography . . . . . . . . . . . .  produce the chapter "Bibliography"
%%
% header for the bibliography
\def\BibHeader{Bibliography}

% produce the chapter bibliography
\def\Bibliography{\EndMatter\Chapter{\BibHeader}

  \begingroup\undoquotes\frenchspacing
  \parindent\manindent
  \pretolerance=5000 %badness allowed for hypenation
  \tolerance=5000    %badness allowed before overfull boxes appear 
  \parskip 1ex plus 0.5ex minus 0.5ex
  \def\begin##1##2{} \def\end##1{}
  \let\newblock=\relax \let\em=\sl
  \inputaux\citein{}{bbl}
  \endgroup
}

%%%%%%%%%%%%%%%%%%%%%%%%%%%%%%%%%%%%%%%%%%%%%%%%%%%%%%%%%%%%%%%%%%%%%%%%%%%%%
%%
%F  \Index  . . . . . . . . . . . . . . . . . .   produce the chapter "Index"
%%
% header for the index
\def\IndexHeader{Index}

% explanation at the beginning of the index
\def\IndexMatter{%
This index covers only this manual.
A page number in {\it italics} refers to a whole section which is devoted to
the indexed subject.  Keywords are sorted with case and spaces ignored,
e.g., ```PermutationCharacter''' comes before ``permutation group''.%
}

% kerning in full index after letter
\def\idxkern{\kern.3em}

% produce the chapter index
\def\Index{\EndMatter%
% get the proper bookmarking function
\global\let\setbookmarkind=\indexbookmark%
\Chapter{\IndexHeader}

  \IndexMatter
  \bigskip
  \begindoublecolumns
  \pretolerance=5000 %badness allowed for hypenation
  \tolerance=5000    %badness allowed before overfull boxes appear 
  \parindent 0pt \parskip 0pt \rightskip 0pt plus2em \emergencystretch 2em
  \everypar{\hangindent\smallmanindent}
  \def\par{\endgraf\leftskip 0pt}
  \def\sub{\advance\leftskip by\smallmanindent}
  \def\subsub{\advance\leftskip by2\smallmanindent}
  \obeylines
  \inputaux\indexin{}{ind}
  \enddoublecolumns
}

% pseudo chapters used for authors, preface, copyright &c.
\def\PseudoInput#1#2#3{%
\vfill\eject% to ensure the pseudo-chapter no. doesn't get into headline
\advance\chapno by1\secno=0\subsecno=0
\immediate\write\secindout{C #1.tex \thechapter. #2}
\label{#2}\label{#3}} % we do this for the benefit of the HTML manuals

% some often-used LaTeX functions
\def\frac#1#2{{{#1}\over{#2}}}

% as the etalchar gets written out in manual.lab, better make it nonfancy
% (otherwise something breaks).
\def\etalchar#1{$^{#1}$}
\def\etalchar#1{#1}

% nasty trick to cope with the `newcommand' created by bibtex
\def\newcommand#1#2#3#4#5{}

% URL stuff
\def\URL#1{\par\kernttindent\hbox{`#1'}\ifx\EndMatter\empty\else\qquad\fi}
\def\Mailto#1{\penalty-1000\hskip 0pt plus10cm\hbox{`#1'}}

% ragged bottom will avoid large blank spaces
\raggedbottom
\frenchspacing
\vfuzz=2pt
